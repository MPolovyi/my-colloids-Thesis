%!TEX root = ..\MainFile.tex
\chapter{Аналитические модели групповой динамики}
\label{ch:TheoreticalModels}

Теоретический подход к описанию групповых явлений был впервые предложен в 2000 году Tu~\cite{tu2000} и был основан на соображениях симметрии. Будучи развитым Bertin'ом~\cite{bertin2006}, этот подход привел к получению уравнений движения, подобных уравнениям Навье-Стокса (подробнее см.~\ref{sec:ClassicalModelsWIthViscosity}).

Адресуя вопрос о допустимости такого подхода, группой ученых был предложен иной подход~\cite{ratushnaya2007,chepizhko2013}, основанный на статистическом описании системы через функционал микроскопической фазовой плотности. Разница между этими подходами заключается в наличии вязкости в уравнениях первой группы (раздел~\ref{sec:ClassicalModelsWIthViscosity}), но при этом в уравнениях второй группы (раздел~\ref{sec:KulinskyModel}) вязкий член не появляется.

Рассмотрим далее основные постулаты и уравнения двух теорий.

\section{Классические модели} % (fold)
\label{sec:ClassicalModelsWIthViscosity}
	
% section ClassicalModelsWIthViscosity (end)

%!TEX root = ..\MainFile.tex
\section{Модель микроскопической фазовой плотности} % (fold)
\label{sec:KulinskyModel}

	Подход, основанный на рассмотрении функционала микроскопической фазовой плотности, предложенный в последне годы \cite{chepizhko2013,kulinskii2012}, позволил получить, в приближении идеальной самодвижущейся жидкости, уравнения движения, отличные от рассмотренных в разделе \ref{sec:ClassicalModelsWIthViscosity}.

	Строгое получение этих уравнений выходит далеко за рамки работы, потому приведем здесь только исходные положения и результаты.

	Функционал микроскопической фазовой плотности записывается следующим образом:
	\begin{equation}
		N(x,t) = \sum_i \delta(x-x_i(t))
	\end{equation}
	где $x = (\boldsymbol{r}, \boldsymbol{v})$, подчинаяется закону сохранения
	\begin{equation}
		\partial_t N  + \boldsymbol{v} \partial_{\boldsymbol{r}} N + \partial_{\boldsymbol{v}} (\boldsymbol{\dot{v}} N) = 0
	\end{equation}

	Соответствующие плотность и поток плотности получаются как
	\begin{align}
		\rho^{(m)}(\boldsymbol{r},t) = \int N(x,t)d \boldsymbol{v}
		\\
		\boldsymbol{j}^{(m)}(\boldsymbol{r},t) = \int \boldsymbol{v} N(x,t)d \boldsymbol{v}
	\end{align}

	В контексте модеил Vicsek'а, уравнение движения без шумового члена записывается как
	\begin{equation}
		\frac{d \boldsymbol{v_i}}{dt} = \boldsymbol{\omega_i} \times \boldsymbol{v_i}
	\end{equation}
	и отображает тот факт, что энергия частиц в алгоритме Vicsek'a сохраняется.

	Путем не сложных, но весьма громозких преобразований, для предельного случая идеальной самодвижущейся жидкости можно получить следующее уравнение движения:

	\begin{equation}
	\label{eq:KulinsciiEqOfMotion}
		\frac{\partial v}{\partial x} = -\frac{1}{2 \rho} \frac{\partial}{\partial x} (\rho -\rho v^2)+(\frac{\rho}{2} - \frac{\alpha}{\rho} \frac{\partial v}{\partial x})v + \frac{\beta}{\rho} \frac{\partial v}{\partial x} + \frac{\rho}{2} + \frac{\alpha}{2}) v^3
	\end{equation}

	Полученное уравнение показывает, что самодвижущиеся жидкости ведут себя отлично от молекулярных жидкостей, и даже от жидкостей, состоящих из активных броуновских частиц.

	Сравнение же этих результатов с рассмотренными в разделе \ref{sec:ClassicalModelsWIthViscosity}, как уже говорилось, приводит к возникновению вопроса о существовании вязкости в модели Vicsek'a

% section KulinskyModel (end)
