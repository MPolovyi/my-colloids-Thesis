%!TEX root = ..\MainFile.tex
\section{Модель микроскопической фазовой плотности} % (fold)
\label{sec:KulinskyModel}

	Подход, основанный на рассмотрении функционала микроскопической фазовой плотности, предложенный в последне годы~\cite{chepizhko2013,kulinskii2012}, позволил получить, в приближении идеальной самодвижущейся жидкости, уравнения движения, отличные от рассмотренных в разделе~\ref{sec:ClassicalModelsWIthViscosity}.

	Строгое получение этих уравнений выходит далеко за рамки работы, потому приведем здесь только исходные положения и результаты.

	Функционал микроскопической фазовой плотности записывается следующим образом:
	\begin{equation}
		N(x,t) = \sum_i \delta(x-x_i(t))
	\end{equation}
	где $x = (\boldsymbol{r}, \boldsymbol{v})$, подчинаяется закону сохранения:
	\begin{equation}
		\partial_t N  + \boldsymbol{v} \partial_{\boldsymbol{r}} N + \partial_{\boldsymbol{v}} (\boldsymbol{\dot{v}} N) = 0
	\end{equation}

	Соответствующие плотность и поток плотности получаются как:
	\begin{align}
		\rho^{(m)}(\boldsymbol{r},t) = \int N(x,t)d \boldsymbol{v}
		\\
		\boldsymbol{j}^{(m)}(\boldsymbol{r},t) = \int \boldsymbol{v} N(x,t)d \boldsymbol{v}
	\end{align}

	В контексте модеил Vicsek'а уравнение движения без шумового члена записывается как:
	\begin{equation}
		\frac{d \boldsymbol{v_i}}{dt} = \boldsymbol{\omega_i} \times \boldsymbol{v_i}
	\end{equation}
	и отображает тот факт, что энергия частиц в алгоритме Vicsek'a сохраняется.

	В наиболее общей форме, уравнение для поля гидродинамической скорости $\boldsymbol{v}$ получается из уравнения для одночастичной функции распределения:
	\begin{equation}
		(\partial_t + \boldsymbol{v} \partial_{\boldsymbol{r}}+\partial_{\boldsymbol{v}}\boldsymbol{\mathcal{F}}) f_1(x,t) = I(x,t)
	\end{equation}
	где накладывается ограничение
	\begin{equation}
		\int \delta v_i f_1 dx= 0
	\end{equation}

	и записывается следующим образом:
	\begin{equation}
	\label{eq:KulinskiMainEqation}
		\frac{\partial \rho v_i}{\partial t} + \frac{\partial}{\partial x_i} = -\frac{\partial P_{ij}}{\partial x_j} + \mathcal{F}_i + n \int \delta v_i I(x,t)d\boldsymbol{v}
	\end{equation}

	В уравнении~\ref{eq:KulinskiMainEqation} $P_{i,j}$ --- тензор давления, а $\mathcal{F}$ --- средняя плотность силы самосогласованного поля.

	Переходя сначала к гидродинамическому пределу идеальной жидкости вичековского типа, то есть пренебрегая всеми вкладами корелляций в уравнении~\ref{eq:KulinskiMainEqation}, можно получить следующее базовое уравнение:
	\begin{equation}
		\frac{d \boldsymbol{v}}{dt} = -\frac{\nabla p_0}{\rho} + p_0 \boldsymbol{v}
	\end{equation}
	где $p_0 = \frac{\rho}{2} (1-\boldsymbol{v}^2)$, для которого можно получить решение в виде, независимом от пространственной координаты:

	\begin{equation}
	 	\rho = \rho_0, \quad |\boldsymbol{v}(v)| = \frac{1}{\sqrt{1+e^{-\rho_0 t}(\frac{1}{u_0^2}-1)}}
	\end{equation} 

	Учет шума может быть проведено путем рассмотрения перехода от предела нулевого шума к пределу низкого шума. Тогда становится возможным использовать приближение локального равновесия.

	В таком приближении для одномерного случая можно разложить тензор давления в ряд, и путем достаточно громоздких преобразований, также пользуясь результатами~\cite{kulinskii2009}, можно выписать уравнение движения системы самодвижущихся частиц для одномерного случая:

	\begin{equation}
	\label{eq:KulinsciiEqOfMotion}
		\frac{\partial v}{\partial x} = -\frac{1}{2 \rho} \frac{\partial}{\partial x} (\rho -\rho v^2)+(\frac{\rho}{2} - \frac{\alpha}{\rho} \frac{\partial v}{\partial x})v + (\frac{\beta}{\rho} \frac{\partial v}{\partial x} + \frac{\rho}{2} + \frac{\alpha}{2}) v^3
	\end{equation}

	Можно сравнить коэффициенты $\mu$ и $\varepsilon$ с результатами~\cite{bertin2006}.

	В приближении этого раздела, 
	\begin{equation}
	\label{eq:KulinsciiCoefficients}
		\mu = \frac{1}{2}\rho, \quad \xi = \frac{1}{2\rho} + \frac{\alpha}{2\rho^2}
	\end{equation}
	в то время как в~\cite{bertin2006} эти коэффициенты равны
	\begin{equation}
		\mu = \frac{8 \rho}{3\pi} \approx 0.85 \rho, \quad \xi = \frac{4}{\pi\rho} - \frac{6 \sigma^2}{\pi \rho} + \mathcal{O}(\sigma^4)
	\end{equation}

	Полученное уравнение показывает, что самодвижущиеся жидкости ведут себя отлично от молекулярных жидкостей и даже от жидкостей, состоящих из активных броуновских частиц.

	Сравнение же этих результатов с рассмотренными в разделе~\ref{sec:ClassicalModelsWIthViscosity}, как уже говорилось, приводит к возникновению вопроса о существовании вязкости в модели Vicsek'a, поскольку в этом подходе показывается отсутствие сдвиговой вязкости, в то время как в других приближениях вклад вязкости проявлялся в гидродинамических уравнениях, и был не равен нулю даже при стремлении шума к нулю.