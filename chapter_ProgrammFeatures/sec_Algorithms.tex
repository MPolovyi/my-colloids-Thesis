\subsection{Описание программы, моделирующей модель Вичека} % (fold)
\label{sub:ViksecModelProgramm}
    Итак, как должно быть понятно из пункта \ref{sub:MultithreadMulticoreDataProcessing}, мы полагаем, что модель Вичека, а конкретно ее особенности связанные с локальностью взаимодействия, и архитектура GPU отлично подходят друг для друга.

    Наконец, алгоритм вычисления в нашем случае выглядит следующим образом: []
    Поскольку задача оптимизации вычислений ставилась лишь косвенно, мы не экспериментировали с тем, что касается разбиения задачи на потоки, и с частотой сортировки информации в памяти, и возможно было бы достичь лучшей оптимизации.
    
    Однако, предложено использовать простой критерий релаксации системы. Как известно из термодинамики, влияние флуктуаций на параметры системы пропорционально $1/\sqrt{N}$, где $N$ - число частиц. Потому мы предложили выполнять для начала 100 шагов по времени для каждого значения шума, а затем, если разница средней скорости и средней скорости на предыдущей итерации \textit{НЕ} оказывается меньше чем $1/\sqrt{N}$, удваивать число шагов, и продолжать симуляцию с тем же значением шума (см. алгоритм на рис. [])\marginpar{нарисовать алгоритм!}
% subsection ViksecModelProgramm (end)

\subsection{Получение профилей скорости Куэтт-подобного потока} % (fold)
\label{sub:SpeedProfilesOfCouetteFlow}
    Как было сказано в главе \ref{ch:TheoreticalModels}, посвященном теоретическим моделям самодвижущихся частиц, признанные широкой общественностью модели самодвижущихся частиц на данный момент так или иначе включают в себя вязкость. Однако не сложно заметить, что в модель Вичека, ввиду отсутствия в ней взаимодействия между частицами, отличного от упорядочивания направления движения, не предполагает наличия между частицами вязких сил в привычном понимании. Известно, что простейшей задачей связанной с течением вязкой жидкости, допускающей к тому-же аналитическое решение, является течение Куэтта. В таком случае установившаяся скорость потока в зависимости от высоты описывается следующим выражением:
    \begin{equation} \label{eq:CoetteFlow}
        u(y) = u_0 \frac{y}{h} + \frac{1}{2\mu} + (\frac{dp}{dx}(y^2 -hy))
    \end{equation}
    где $\mu$ - вязкость жидкости, $\frac{dp}{dx}$ - градиент давления. Результирующие профили скорости представлены на рис [].
    Таким образом, для обоснования применимости теории, представленной в \cite{chepizhko2013} и рассмотренной в пункте \ref{sec:KulinskyModel}, достаточно показать, что установившиеся профили скорости в Куэтт-подобной задаче для самодвижущихся частиц никоим образом не соотносятся с профилями течения Куэтта вязкой жидкости.

    Задача Куэтта в классическом понимании ставится следующим образом: вязкая ньютоновская жидкость, заключенна между двумя бесконечными горизонтальными плоскостями на расстоянии $h$ одна от другой, нижняя пластина покоится (в силу вязкости скорость жидкости $u(0) = 0$), в то время как верхняя движется в направлении $x$ со скоростью $u_0$ (аналогично, в силу вязкости скорость жидкости $u(h) = u_0$). Необходимо найти зависимость скорости жидкости от высоты.

    Поскольку мы постулируем отсутствие вязкости, и тем более потому что речь идет о самодвижущихся частицах, нобходимо несколько модифицировать граничные условия. Для начала, рассматривается прямоугольная область, на вертикальных стенках которой заданы периодические граничные условия. Потом, нижняя стенка принята отражающей, что конечно не совсем верно и приводит к некоторым особенностям полученных профилей. (см. главу \ref{ch:Results}). На верхней границе принято следующее: что бы не нарушать допущение об отсутствии взимодействий кроме меняющих направление, верхняя граница учитывается как дополнительное ``скопление частиц'', движущееся в заданном направлении, и учитывается после учета взаимодействия частицы с окружением, в том случае если расстояние до нее оказывается меньше радуиса взаимодействия частиц.
% subsection SpeedProfilesOfCouetteFlow (end)