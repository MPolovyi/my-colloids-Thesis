%!TEX root = ..\MainFile.tex
\section{Задача, поставленная перед численным моделированием} % (fold)
\label{sec:MotivationForProgramm}
    Как мы рассмотрели в главе~\ref{ch:TheoreticalModels}, на данный момент существует два основных теоретических подхода к описанию группового движения. Важным отличием между ними является наличие (или отсутствие) вязкости. Поскольку обе модели являются приближенными и к тому же допускают решение только в ограниченном числе случаев, прямое сравнение результатов с экспериментом провести достаточно затруднительно. Однако проведение эксперимента само по себе не является проблемой - как указано в главе~\ref{ch:ComputerModelsOfHords}, компьютерное моделирование является основным способом изучения систем, демонстрирующих групповую динамику.

    Таким образом, поставив численный эксперимент над системой с граничными условиями, в которой обязательно проявляется вязкость, можно получить ответ на главный вопрос - присутствует ли вязкость в самодвижущейся жидкости?

    Такого рода системой было выбрано течение Куэтта, несколько адаптированное к реалиям самодвижущихся частиц, см. раздел~\ref{sec:KuetteAdaptation}.

    Прежде чем перейти к описанию работы приложения, необходимо сделать несколько замечаний, которые способны существенно сократить время симуляций:

    \begin{itemize}
        \item в силу детерменированности взаимодействия частиц в модели Vicsek'a возможно выполнение симуляции в несколько потоков
        \item простота модели и очевидная локальность взаимодействия позволяет при помощи специального размещения данных в памяти существенно сократить число вычислений
        \item кроме того, простые арифметические действия, из которых состоят уравнения движения~\ref{eq:ViksecEquationsOfMotion}, позволяют перенести вычисления на GPU, используя CUDA-подобную технологию C++ AMP
        \item для формирования устоявшегося состояния системы из статистической механики известно, что достаточно, чтобы флуктуации параметра порядка были $< \frac {1}{\sqrt{N}}$, где $N$ - число частиц. Это может быть использовано в качестве критерия для прекращения симуляции.
        \end{itemize}

    Подробнее эти вопросы будут рассмотрены в разделе~\ref{sec:ProgrammOptimisations}
% section MotivationForProgramm (end)