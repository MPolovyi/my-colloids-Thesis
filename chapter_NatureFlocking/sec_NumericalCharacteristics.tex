\section{Количественные характеристики группового движения} % (fold)
\label{sec:NumericalCharacteristicsCollMot}
    Не смотря на то, что подобность движений различных типов обьектов очевидна, для какого-либо исследования нельзя обойтись без количественных характеристик явления.
    Укажем отдельно характеристики обьектов, которые участвуют в коллективном движении:
    \begin{enumerate}
        \item все они похожи друг на друга
        \item они перемещаются с почти постоянной скоростью и способны изменять направление движения
        \item они взимодействуют когда находятся на расстоянии меньшем эффективного, и посредством этого изменяют направление движения в стремясь к выравниванию
        \item направление движения подвержено шуму определенной амплитуды
    \end{enumerate}

    Рассматривая же обьекты в совокупности, необходимо заметить, что при определенных условиях происходит (скачкообразный) переход от разупорядоченного к упорядоченному движению. Такое явление называется фазовым переходом. Наиболее знакомые всем нам фазы, в которых может пребывать вещество: газообразная, жидкая и твердая --- переходят одна в другую при изменении как минимум одного внешнего параметра --- температуры. При этом плотность выступает в роли {\itпараметра порядка} --- некоторой величины, которая однозначно характеризует фазу вещества.

    Пользуясь этой аналогией и помня о том, что изменяется только направление движения каждого обьекта, в качестве параметра порядка было естественно определить следующей величиной:
    \begin{equation}\label{eq:OrederParam}
        \varphi = \frac{1}{N v_0} |{\sum\limits_{i=1}^n \vec{v}_i}|
    \end{equation}
    Параметром порядка, согласно этому определению, является модуль средней скорости, нормированной на единицу. В уравнении \ref{eq:OrederParam} $N$ --- полное число обьектов, $v_0$ --- средний модуль скорости обьектов в системе. Понятно, что если движение неупорядочено, то скорости обьектов направлены случайным образом, и такая сумма будет стремиться к нулю. Если же скорости направлены вдоль некоторого выбранного направления, то параметр порядка стремится к единице. Разумеется, как и всяческий статистический параметр, выражаемый через средние величины, это выражение имеет смысл только при больших $N$, иначе величина флуктуаций из-за кадого обьекта будет слишком высокой.

    Из статистической физики нам известно, что фазовый переход может происходить как в равновесной (замкнутой), так и в неравновесной системе. Ввиду того что в групповом движении, согласно рассматриваемой модели, участвуют самодвижущиеся частицы становится понятно, что системы с групповой динамикой сугубо неравновесны. Неравновесная статистическая физика в последнее десятилетие выделилась в отдельное направление физики, со своим словарем, обьектом исследований и характерными процессами, вызывающими наибольший интерес исследователей в текущий момент времени. И потому, возможно, было бы предпочтительнее модифицировать модель или провести аналогии с фазовыми переходами, рассматриваемыми в рамках равновесной статистической физики.~\cite{vicsek2012}

    Эта идея, однако, кажется не приемлимой по нескольким соображениям: во-первых, в системах, демонстрирующих групповую динамику, наблюдаются совершенно не характерные для равновесной термодинамики явления, такие как ``пробки'' или флуктуации гигантского числа частиц []. А во-вторых, статистическая механика имеет дело с количеством частиц, стремящимся к бесконечности ($10^{23}$), и в рамках этого допущения определяет остальные понятия, в то время как явления группового перемещения наблюдаются при числе частиц, редко превышающем десятки тысяч.

    Интересным было замечание [], указывающее на значительную аналогию в поведении систем, демонстрирующих групповую динамику, и феромагнитных систем. К примеру, рассматривая критические экспоненты, связанные с параметром порядяка вида:
    \begin{equation}
        \sigma \sim |1-\eta/\eta_c|^{-\gamma}
    \end{equation}
    позволяют вычислить критические индексы $\gamma$. При этом получается, что критические индексы вычисленные, скажем, по модели Изинга, и вычисленные из (численных) экспериментов с системами, демонстрирующими групповую динамику, оказываются в хорошем согласии между собой. За более полной информацией о параллелях между феромагнетиками и системами с групповой динамикой смотри []\marginpar{doi:10.1088/0305-4470/30/5/009}.

    Итак, резюмируя, в системах с коллективными эффектами происходит фазовый переход, при этом параметром порядка выступает средняя векторная скорость \ref{eq:OrederParam}, а в качестве ``температуры'' --- шум, т.е. некоторое добавочное воздействие, которое случайным образом изменяет направление движения каждого обьекта в рассматриваемой системе. Возникает вопрос --- единственный ли это параметр, который может повлиять на фазу системы и привести к фазовому переходу? Как было экспериментально проверено Камбуи~\cite{cambui2012} и многими другими, фазовый переход в системе взаимодействующих обьектов также возможен при увеличении плотности. Кроме ``обычного'' фазового перехода, при котором изменяется только направление скорости обьектов, иногда наблюдается явление, в котором при увеличении плотности уменьшается скорость перемещения обьектов, и образуется ``пробка''[] \marginpar{Keller-segel,~\cite{keller1971}}
    % subsection NumCharCollMot (end)
    На этом мы заканчиваем, сложность систем требует вдумчивого и основательного подхода, и потому изначально было предложено значительное количество компьютерных моделей~\cite{reynolds1987,vicsek1995,gregoire2004,nagy2007,schubring2013}, и лишь потом, на основании результатов, полученных в численных экспериментах, были разработаны теоретические подходы к групповому движению.~\cite{tu2000,kulinskii2005,bertin2006,ratushnaya2007,vicsek2007,chepizhko2013,kulinskii2014}