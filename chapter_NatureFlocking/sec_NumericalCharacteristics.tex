\section{Количественные характеристики группового движения} % (fold)
\label{sec:NumericalCharacteristicsCollMot}
    Несмотря на то, что подобность движений различных типов обьектов очевидна, для какого-либо исследования нельзя обойтись без количественных характеристик явления.
    Укажем отдельно характеристики обьектов, которые участвуют в коллективном движении:
    \begin{enumerate}
        \item все они похожи друг на друга
        \item они перемещаются с почти постоянной скоростью и способны изменять направление движения
        \item при сближении взаимодействие приводит к выравниванию направления движения
        \item выравнивание происходит с ошибками (шумом)
    \end{enumerate}

    Для совокупности обьектов характерным является наличие фазового перехода от разупорядоченного движения к упорядоченному.

    Помня о том, что изменяется только направление движения каждого обьекта, в качестве параметра порядка было естественно определить следующую величину:
    \begin{equation}\label{eq:OrederParam}
        \varphi = \frac{1}{N v_0} |{\sum\limits_{i=1}^n \vec{v}_i}|
    \end{equation}
    Тогда мерой упорядоченности системы является модуль средней скорости, нормированной на единицу. В уравнении~\ref{eq:OrederParam} $N$ --- полное число обьектов, $v_0$ --- средний модуль скорости обьектов в системе. Понятно, что если движение неупорядочено, то скорости обьектов направлены случайным образом, и такая сумма будет стремиться к нулю. Если же скорости направлены вдоль некоторого выбранного направления, то параметр порядка стремится к единице. Разумеется, как и всяческий статистический параметр, выражаемый через средние величины, это выражение имеет смысл только при больших $N$, иначе величина флуктуаций из-за каждого обьекта будет слишком высокой.

    Из статистической физики нам известно, что фазовый переход может происходить как в равновесной (замкнутой), так и в неравновесной системе. Ввиду того, что в групповом движении, согласно рассматриваемой модели, участвуют самодвижущиеся частицы, становится понятно, что системы с групповой динамикой сугубо неравновесны. Неравновесная статистическая физика в последнее десятилетие выделилась в отдельное направление физики, и потому, возможно, было бы предпочтительнее модифицировать модель или провести аналогии с фазовыми переходами, рассматриваемыми в рамках равновесной статистической физики.~\cite{vicsek2012}

    Эта идея, однако, кажется неприемлимой по нескольким соображениям: во-первых, в системах, демонстрирующих групповую динамику, наблюдаются совершенно не характерные для равновесной термодинамики явления, такие как ``пробки'' или флуктуации гигантского числа частиц. А во-вторых, статистическая механика имеет дело с количеством частиц, стремящимся к бесконечности ($10^{23}$), и в рамках этого допущения определяет остальные понятия, в то время как явления группового перемещения наблюдаются при числе частиц, редко превышающем десятки тысяч.

    Интересным было замечание~\cite{tu2000}, указывающее на значительную аналогию в поведении систем, демонстрирующих групповую динамику, и феромагнитных систем. К примеру, рассмотрение критических экспонент, которые связанны с параметром порядяка соотношениями вида:
    \begin{equation}
        \sigma \sim |1-\eta/\eta_c|^{-\gamma}
    \end{equation}
    позволяет вычислить критические индексы $\gamma$. При этом получается, что критические индексы, вычисленные, скажем, по модели Изинга и определенные из (численных) экспериментов с системами, демонстрирующими групповую динамику, оказываются в хорошем согласии между собой.

    Итак, резюмируя, в системах с коллективными эффектами происходит фазовый переход, при этом параметром порядка выступает средняя векторная скорость~\ref{eq:OrederParam}, а в качестве ``температуры'' --- шум, т.е. некоторое добавочное воздействие, которое случайным образом изменяет направление движения каждого обьекта в рассматриваемой системе. Возникает вопрос --- единственный ли это параметр, который может повлиять на фазу системы и привести к фазовому переходу? Как было экспериментально проверено Камбуи~\cite{cambui2012} и многими другими, фазовый переход в системе взаимодействующих обьектов также возможен при увеличении плотности. Кроме ``обычного'' фазового перехода, при котором изменяется только направление скорости обьектов, иногда наблюдается явление, в котором при увеличении плотности уменьшается скорость перемещения обьектов и образуется ``пробка''.~\cite{keller1971}
    % subsection NumCharCollMot (end)