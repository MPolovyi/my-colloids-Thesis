\section{Методы наблюдения и сбора информации} % (fold)
\label{sec:ExperimentalMethods}
	Главная сложность в выполнении экспериментального наблюдения за группой животных или, тем более, насекомых заключается в трудности отслеживания траекторий отдельных особей. Это связано с тем, что рассматриваемые колонии или группы
	\begin{itemize}
		\item  состоят из множества индивидов, которые
		\item выглядят очень похоже
		\item и в основном очень быстро перемещаются
	\end{itemize}
	Кроме того, необходимо учитывать крайнее разнообразие живых организмов, за перемещением которых необходимо наблюдать, поскольку, как уже говорилось во введении, изменение поведения при обьединении в группы наблюдается для широкого разнообразия типоразмеров животных --- начиная от бактерий в пробирке~\cite{csahok1997,keller1971} и заканчивая китами на океанских просторах~\cite{makris2009}.
	Несмотря на совсем неочевидные методы решения этих проблем, изучению группового поведения на протяжении многих лет посвящалось значительное внимание.

	Условно их можно разделить на две неравные группы: в одну отнесем так или иначе методы с использованием визуальных средства (т.е. фото- и видеокамер в комбинации с различного рода маячками или метками), а в другую --- сравнительно недавно начавшие применяться методы с использованием GPS-систем.

	Классическим может считаться метод измерения скорости по изображениям частиц~\cite{raffel2007}. Обычно он применяется для измерения скорости потока жидкости или газа: в исследуемую среду вводится краситель, состоящий из броуновских частиц, и при помощи наблюдения за перемещением этого вещества возможно определить линии тока в исследуемой среде. Применимо к самодвижущимся частицам метод успешно был применен Czir\'{o}k  et all\cite{csahok1997}, которые при помощи фазово-контрастной микроскопии исследовали групповое движение молекул.

	Когда речь заходит об исследовании стай птиц (стад животных), размер индивидов больше, однако, в отличие от бактерий, область перемещения слабо ограничена. Чтобы обойти эту сложность, например, Becco et all~\cite{becco2006} ограничили движение рыб по высоте, используя плоский аквариум.

	Использование нескольких камер для съемки стай с разных точек в дальнейшем требовало трудоемкой обработки результатов для получения трехмерной картины, но позволило получить мгновенные снимки положения сотен скворцов в стае~\cite{ballerini2008}, однако получить траектории движения не удалось.

	По мере развития технологий стало возможным создать маячки GPS --- достаточно маленькие, чтобы их можно было закрепить на животном или птице, не мешая ему. Ограничением такого подхода является возрастающая стоимость исследования. Были проведены исследования пар~\cite{biro2006,nagy2010} или даже шестерок~\cite{dellariccia2008} тренированных голубей.

	Помимо вышесказанного, в настоящее время увеличение точности сонаров и разработка определенной технологии применения, в которой океан выступает в роли акустического волновода, позволило использовать их для единовременного наблюдения за огромными количествами рыб на больших территориях океанического шельфа.~\cite{makris2006}

	Подводя итоги, можно отметить увеличение роли комьютера в накоплении и обработке экспериментальных данных. Текущая динамика позволяет надеяться, что со временем будут получены данные не меньшей точности, чем получаемые во время численных экспериментов по различным моделям.
% subsection subsection_name (end)