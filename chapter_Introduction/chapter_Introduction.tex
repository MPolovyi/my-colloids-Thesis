%!TEX root = ..\MainFile.tex
\likechapter{Введение}

Считается, что впервые задача моделирования поведения ``разумно'' взаимодействующих объектов была поставлена и удовлетворительно решена К.~Рейнольдсом в 1987 году \cite{reynolds1987}. Его модель носила исключительно прикладной характер, и использовалась для моделирования поведения стай птиц, групп животных или даже космических кораблей при создании анимационных фильмов.

В природе существует безграничное многообразие взаимодействий между живими существами. Однако, многолетние наблюдения показали, что все живые организмы, собираясь в группы, демонстрируют особый класс группового взаимодействия, в результате чего было выделено отдельное направление в биологии, изучающее этот феномен.

Параллельно с этим, в 1995 годы T.~Vicsek'ом и коллегами \cite{vicsek1995} была предложена минимальная модель, описывающаяся простыми уравнениями, и демонстрирующая многие аспекты групповой динамики. Наиболее интересным с физической точки зрения в их модели оказалось наличие явления фазового перехода от разупорядоченного к упорядоченному поведению частиц, называнных ``самодвижущимися''.

С тех пор было предложено некоторое количество аналогичных моделей, к примеру \cite{gregoire2004,schubring2013,kuemmel2013,huepe2008,chate2008,tu2000}, было проведено множество симуляций с различными параметрами систем (наиболее полную информацию можно найти в одной из последних работ T.~Vicsek'a \cite{vicsek2012}, и в приведенном списке литературы), а также был получен богатый экспериметнальный материал, представленный в \cite{keller1971,chowdhury2006,czirok1999,csahok1997,buhl2006,ballerini2008,selous1931,dellariccia2008,biro2006,major1978,cambui2012,makris2006,parrish1997,sinclair1977}, и многих других изданиях.

Однако, несмотря на 20-ти летнюю историю, по прежнему в области описания поведения самодвижущихся частиц остается множество пробелов. К примеру, нет однозначности касательно рода фазового перехода, с аргументами как за первый \cite{gregoire2004,aldana2009}, так и за второй \cite{vicsek1995,czirok1999,huepe2008} род.

Не менее интересным является вопрос об аналитическом описании жидкости самодвижущихся частиц. До сих пор не было получено не-приближеных уравнений движения. Среди полученных результатов выделяются работы Tonner'a и Tu \cite{toner1995,tu2000}, в которых были получены феноменологические уравнения, аналогичные уравнениям Навье-Стокса, а также работы Bertin'a и соавторов \cite{bertin2006}, в которых было рассмотрен Больцмановский подход в приближении парного взаимодействия, и получены уравнения, прямо соответствующие феноменологическим.

Недавно был предложен другой подход, основанный на функционале микроскопической фазовой плотности, в результате которого были получены уравнения, не соответствующие феноменологическим в том, что касалось вязкого взаимодействия, в результате чего встал вопрос о сущестовании вязкости в модели самодвижущихся частиц.