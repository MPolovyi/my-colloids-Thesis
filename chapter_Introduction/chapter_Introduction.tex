%!TEX root = ..\MainFile.tex
\likechapter{ВВЕДЕНИЕ}

Считается, что впервые задача моделирования поведения ``разумно'' взаимодействующих объектов была поставлена и удовлетворительно решена К.~Рейнольдсом в 1987 году \cite{reynolds1987}. Его модель носила исключительно прикладной характер, и использовалась для моделирования поведения стай птиц, групп животных или даже космических кораблей при создании анимационных фильмов.

С другой стороны, интерес к поведению роев насекомых, стай птиц или животных можно проследить как минимум на столетие назад, к примеру \cite{selous1931}. Для исследователей того времени, однако, были доступны лишь простейшие методы наблюдения (а именно, визуальные). Не смотря на это, выводы сделанные из наблюдений позволили выделить перемещения взаимодействующих, интеллектуальных обьектов в отдельный класс динамики систем <<частиц>>, которые позже станут известны как самодвижущиеся частицы.%добавить каких-нибудь ссылок на прикладников

Концепция таких частиц была впервые предложена Т.~Вичеком и коллегами \cite{vicsek1995}. Наиболее интересным с физической точки зрения в их модели оказалось наличие явления фазового перехода от разупорядоченного к упорядоченному поведению частиц, крайне похожего на поведение птичьих стай.
С тех пор было предложено некоторое количество аналогичных моделей, к примеру \cite{gregoire2004,schubring2013,kuemmel2013,huepe2008,chate2008,tu2000}, было проведено множество симуляций с различными параметрами систем (наиболее полную информацию можно найти в одной из последних работ Т.~Вичека \cite{vicsek2012}, и в приведенном списке литературы %переписать часть наименований оттуда!
), а также был получен богатый экспериметнальный материал касательно поведения стай птиц [] %добавить источники!\cite{
, косяков рыб \cite{cambui2012}, были обнаружены проявления групповой динамики у бактерий [] и даже у неорганических систем при определенных условиях [].