%!TEX root = ..\MainFile.tex
\likechapter{Введение}

Считается, что впервые задача моделирования поведения ``разумно'' взаимодействующих объектов была поставлена и удовлетворительно решена К.~Рейнольдсом в 1987 году~\cite{reynolds1987}. Его модель носила исключительно прикладной характер и использовалась для моделирования поведения стай птиц, групп животных или даже космических кораблей при создании анимационных фильмов.

В природе существует безграничное многообразие взаимодействий между живими существами. Однако многолетние наблюдения показали, что все живые организмы, собираясь в группы, демонстрируют особый класс группового взаимодействия, в результате чего было выделено отдельное направление в биологии, изучающее этот феномен.

Параллельно с этим, в 1995 годы T.~Vicsek'ом и коллегами~\cite{vicsek1995} была предложена минимальная модель, описывающаяся простыми уравнениями и демонстрирующая многие аспекты групповой динамики. Наиболее интересным с физической точки зрения в их модели оказалось наличие явления фазового перехода от разупорядоченного к упорядоченному поведению частиц, называнных ``самодвижущимися''.

С тех пор было предложено большое количество аналогичных моделей, к примеру~\cite{gregoire2004,schubring2013,kuemmel2013,huepe2008,chate2008,tu2000}, проведено множество симуляций с различными параметрами систем (наиболее полную информацию можно найти в одной из последних работ T.~Vicsek'a~\cite{vicsek2012} и в приведенном списке литературы), а также был получен богатый экспериметнальный материал, см. например в~\cite{keller1971,chowdhury2006,czirok1999,csahok1997,buhl2006,ballerini2008,selous1931,dellariccia2008,biro2006,major1978,cambui2012,makris2006,parrish1997,sinclair1977}.

Однако, несмотря на 20-ти летнюю историю, по-прежнему в области описания поведения самодвижущихся частиц остается множество пробелов. К примеру, нет однозначности касательно рода фазового перехода, так, в работах~\cite{gregoire2004,aldana2009} приводятся аргументы в пользу фазового перехода 1-го рода, в то время как в~\cite{vicsek1995,czirok1999,huepe2008} обосновывается его непрерывный характер (ф. п. 2-го рода).

Не менее интересным является вопрос об аналитическом описании гидродинамики системы самодвижущихся частиц.  Среди полученных результатов выделяются работы Tonner'a и Tu~\cite{toner1995,tu2000}, в которых были получены феноменологические уравнения, аналогичные уравнениям Навье-Стокса, а также работы Bertin'a и соавторов~\cite{bertin2006}, в которых был рассмотрен Больцмановский подход в приближении парного взаимодействия и получены уравнения, прямо соответствующие феноменологическим.

Недавно был предложен другой подход, основанный на функционале микроскопической фазовой плотности, в результате которого было получено уравнение идеальной жидкости вичековского типа (о нем ниже), которое является аналогом уравнения Эйлера для молекулярных жидкостей.