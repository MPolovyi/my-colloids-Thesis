\documentclass[a4paper,14pt,russian]{extreport}

  \usepackage{extsizes}
  \usepackage{titlesec}
  \usepackage{cmap}
  \usepackage[T2A]{fontenc}
  \usepackage[utf8]{inputenc}
  \usepackage[russian]{babel}

  \usepackage{pscyr}
  \usepackage{graphicx}
  \usepackage{amssymb,amsfonts,amsmath,amsthm}
  \usepackage{indentfirst}
  \usepackage[usenames,dvipsnames]{color}
  \usepackage{makecell}
  \usepackage{multirow}
  \usepackage{ulem}
  \usepackage{tocloft}
  \usepackage{import}
  \usepackage{lastpage}
  \usepackage{etoolbox}
  \usepackage[title,titletoc]{appendix}
  \usepackage{pdfpages}
  \usepackage{listings}

  \usepackage[tableposition=top]{caption}
  \usepackage{subcaption}
  \DeclareCaptionLabelFormat{gostfigure}{Рисунок #2}
  \DeclareCaptionLabelFormat{gosttable}{Таблица #2}
  \DeclareCaptionLabelSeparator{gost}{~---~}
  \captionsetup{labelsep=gost}
  \captionsetup[figure]{labelformat=gostfigure}
  \captionsetup[table]{labelformat=gosttable}
  \renewcommand{\thesubfigure}{\asbuk{subfigure}}

  \linespread{1.3}
  \renewcommand{\rmdefault}{ftm}
  \frenchspacing

  \titleformat{\chapter}[display]
    {\filcenter}
    {\MakeUppercase{\chaptertitlename} \thechapter}
    {8pt}
    {\bfseries}{}
    
  \titleformat{\paragraph}[display]
    {\filcenter}
    {\MakeUppercase{\chaptertitlename} \thechapter}
    {8pt}
    {\bfseries}{}
    
  \titleformat{\section}
    {\normalsize\bfseries}
    {\thesection}
    {1em}{}
    
  \titleformat{\subsection}
    {\normalsize\bfseries}
    {\thesubsection}
    {1em}{}
    
  \titlespacing*{\chapter}{0pt}{-30pt}{8pt}
  \titlespacing*{\paragraph}{0pt}{-30pt}{8pt}
  \titlespacing*{\section}{\parindent}{*4}{*4}
  \titlespacing*{\subsection}{\parindent}{*4}{*4}

  \renewcommand{\cfttoctitlefont}{\hspace{0.38\textwidth} \bfseries\MakeUppercase}
  \renewcommand{\cftbeforetoctitleskip}{-1em}
  \renewcommand{\cftaftertoctitle}{\mbox{}\hfill \\ \mbox{}\hfill{\footnotesize Стр.}\vspace{-2.5em}}
  \renewcommand{\cftchapfont}{\normalsize\bfseries \MakeUppercase{\chaptername} }
  \renewcommand{\cftsecfont}{\hspace{31pt}}
  \renewcommand{\cftsubsecfont}{\hspace{11pt}}
  \renewcommand{\cftbeforechapskip}{1em}
  \renewcommand{\cftparskip}{-1mm}
  \renewcommand{\cftdotsep}{1}
  \setcounter{tocdepth}{2}
  \setcounter{secnumdepth}{5}

  \usepackage[square,numbers,sort&compress]{natbib}
  \newcounter{totfigures}
  \newcounter{tottables}
  \newcounter{totreferences}
  \makeatletter
  \renewcommand{\@dotsep}{2}
  \newcommand{\l@likechapter}[2]{{\bfseries\@dottedtocline{0}{0pt}{0pt}{#1}{#2}}}
  \AtEndDocument{%
    \addtocounter{totfigures}{\value{figure}}%
    \addtocounter{tottables}{\value{table}}%
    \immediate\write\@mainaux{%
      \string\gdef\string\totfig{\number\value{totfigures}}%
      \string\gdef\string\tottab{\number\value{tottables}}%
      \string\gdef\string\totref{\number\value{totreferences}}%
    }%
  }
  \newcommand{\append}[1]{
    \clearpage
    \stepcounter{chapter} 
    \paragraph{\MakeUppercase{#1}}
    \empline
    \addcontentsline{toc}{likechapter}{\texorpdfstring{\MakeUppercase{\chaptertitlename~\Asbuk{chapter}\;#1}}{\chaptertitlename~\Asbuk{chapter}:~#1}}}
    
  \usepackage{geometry}
  \geometry{left=3cm}
  \geometry{right=1.5cm}
  \geometry{top=2.4cm}
  \geometry{bottom=2.4cm}

  \usepackage[pdftex,bookmarks=true,colorlinks=true,linkcolor=blue,citecolor=Green,linktoc=none]{hyperref}
  \newcommand{\includechapter}[1]{\subimport{chapter_#1/}{chapter_#1}}

  \bibliographystyle{gost705}
  \usepackage{epigraph}
  \epigraphwidth 200pt
  \usepackage{wrapfig}
  \usepackage{float}
  % for figures: caption label is italic, the caption text is bold / italic
  \captionsetup[figure]{labelfont={bf,it},textfont=it}
  % for subfigures: caption label is bold, the caption text normal.
  % justification is raggedright (i.e. left aligned)
  % singlelinecheck=off means that the justification setting is used even when the caption is only a single line long. 
  % if singlelinecheck=on, then caption is always centered when the caption is only one line.
  \captionsetup[subfigure]{labelfont=bf,textfont=normalfont,singlelinecheck=off,justification=raggedright}

\title{\LaTeX}
\date{}
\author{}
 
\begin{document}
  Добрый день, меня зовут Полевой Максим, тема моей работы - ``Компьютерное моделирование идеальной самодвижущейся жидкости по модели Vicsek'a''

  План моего доклада таков:
  \begin{itemize}
    \item Сначала я представлю вам цель работы
    \item Потом я приведу некоторые примеры группового движения в природе
    \item После этого я напомню (?) основные положения модели Вичека, и предложенные его им и его группой критерии для количественного описания групповых движений
    \item Далее я вкратце рассмотрю преобладающий сейчас подход к теоретическому описанию самодвижущейся жидкости
    \item И поставлю его под сомнение
    \item В связи с чем будет рассмотрен новейший подход к теоретическому описанию, основанный на функционале микроскопической фазовой плотности
    \item И поставлен вопрос об определении вязкости иедальной самодвижущейся жидкости
    \item Для этого будет проведено прямое компьютерное моделирование (по уравнениям Вичека) Куэттовского потока
    \item Попутно будет охвачен вопрос об оптимизации этой симуляции
    \item И после приведения результатов (т.е. профилей скорости Куэттовского потока)
    \item Будут сделаны выводы о ``наиболее правильном'' подходе к теоретическому описанию идеальной самодвижущейся жидкости
  \end{itemize}

  Итак, поехали. Целью моей работы является ``Путем прямой симуляции потока Куэтта самодвижущейся жидкости по модели Vicsek’a ответить на вопрос о состоятельности аналитических моделей самодвижущейся жидкости, в которых фигурирует вязкость.
''
  
  ... слайд 4 ...

  Все мы хоть раз в жизни наблюдали или за полетом стаи птиц, или видели как они, вдруг, взмывают в воздух. Видели как стаи уток плавают в парке в поисках хлеба, 

  ... слайд 5 ...

  или в том-же парке рыбы то беспорядочно мечутся из стороны в сторону, то наоборот сбиваются в косяки и плывут в одном направлении. Все это - феномен группового движения. Оно имеет место на всех масштабах природы, 

  ... слайд 6 ...

  начиная от бактерий в пробирке, и заканчивая китами на океанских просторах.
  
  Вичек и соавторы в 1995 году сделали первый шаг в количественном изучении группового движения.

  ... слайд 7 ...

  Ими была предложена модель точечных частиц, движущихся с постоянной скоростью и способных изменять направление своего движения учитывая направление движения частиц в радиусе взаимодействия. (Результаты симуляции модели можно видеть на рисунке). Необходимо обратить внимание на качественное различие между первым и третьим изображением - так в модели провяляется фазовый переход.

  Этот фазовый переход, очевидно, нарушает теорему Мермина-Вагнера (о том, что невозможно спонтанное нарушение симметрии при конечной температуре в пространстве размерностью $d <= 2$), и позволяет заключить что, 

  ... слайд 8 ...

  во-первых, в пределе бесконечно малой скорости модель Вичека аналогична модели феромагнетика, а во-вторых, что более важно, стаи не подчиняются законам равновесной термодинамики, и дальний порядок связи, очевидно наблюдаемый на предыдущем слайде, является результатом нелинейной реакции на локальные флуктуации (шум). Но я несколько отклонился от сути.

  Взаимодействие между частицами в модели Вичека локально - направление движения частиц каждый раз изменяется так, чтобы совпадать со средним направлением движения частиц в радиусе взаимодействия (показать на рисунке).


  ... слайд 9 ...

  Неоспоримым достоинством модели Вичека является ее простота. Для описания поведения части достаточно всего двух простымх уравнениямй, одно из которых описывает изменение скорости частицы в зависимости от средней скорости движения частиц в радиусе взаимодействия, а другое определяет способ введени в модель шума - левая сторона соответствует т.н. векторному шуму, а права - т.н. скалярному. Следует отметить, что от качества шума зависит тип фазового перехода, упомянутого ранее.

  Однако несмотря на простоту в реализации и описании, за годы исследований было показано, что варьируя параметры модели Вичека - скорость движения частиц, плотность частиц, метод введения шума и тому подобное, можно получить очень широкое разнообразие т.н. паттернов групового движения, и именно поэтому эта модель всегда принимается в качестве базовой.

  ... слайд 10 ...

  До 2000 года развитие учения (?) о групповом движении шло по пути компьютерного изучения и параллельно модернизации модели Вичека. Однако в 2000 году Ту, (а затем к нему присоединился Тоннер), представил непрерывное по времени уравнение, полученное феноменологически, из соображений симметрии, описывающее динамику самодвижущихся частиц. 

  Необходимо обратить внимание, что поскольку система не является Галилей-инвариантной, то в отличие от уравнений Н-С, сохраняются все коэффициенты $\lambda$, а не только $\lambda_1$

  ... слайд 11 ...

  Подход, предложенный Тоннером и Ту получил дальнейшее развитие в работах Бертина и соавторов, которые, отталкиваясь от уравния Больцмана, и учитывая специфику взаимодействия между частицами, вывели уравнение движения, а также получили явное выражение для коэффициента вязкости. Следует отметить, что их уравнение по своей сути соответствует феноменологическому уравнению.

  ... слайд 12 ...

  Однако, возникают определенные сомнения в справедливости подхода, постулирующего вязкость, для идеальной самодвижущейся жидкости. Лучше всего рассмотреть эти сомнения на примере парного взаимодействия молекул Ньютоновской и самодвижущейся жидкостей.

  Для ньютоновской жидкости столкновения, в силу равновероятности всех направлений движения частиц, приводят к распределению Максвела, но что более важно, именно на соударения частиц, в результате которых происходи увеличение энтропии системы (из-за уменьшения упорядоченности движения броуновских частиц). С другой стороны, взаимодействие идеальных самодвижущихся частиц из-за их особого типа взаимодействия приводит, наоборот, к увеличению упорядоченности системы. А из статистической физики нам известно, что при этом энтропия системы убывает, а значит, можно считать идеальную самодвижущуюся жидкость как минимум не диссипативной.

  ... слайд 13 ...

  Поэтому сравнительно недавно некоторыми исследователями (и в стенах нашего факультета тоже) был предложен и разработан подход, основанный на функционале микроскопической фазовой плотности. В результате этого подхода, в приближении идеальной жидкости, было получено уравнение движения. Его наиболее характерным отличием является отсуствие в нем члена, ответственного за вязкость.

  ... слайд 14 ...

  Для того чтобы разрешить противоречие между двумя подходами, нами было предложено провести прямое моделирование Куэтт-подобного течения, по уравнениям предложенным Вичеком, с тем чтобы по полученным профилям скорости можно было судить о наличии или отсутствии вязкости.

  Как известно, для ньютоновской жидкости профилем  скорости является наклонная прямая при отсутствии градиента давления, и выпуклая кривая сложной формы при наличии градиента давления.

  ... слайд 15 ...

  Задача Куэтта должна быть несколько адаптирована для самодвижущейся жидкости. Поскольку мы не можем заставить частицы терять скорость на нижней границе, нами было рассмотрено два варианта: в обоих из них верхняя граница учитывалась как указано на рисунке, но в одном варианте нижняя грань выступала в роли зеркального отражения, а в другом на нижней грани угол отражения был меньше угла падения, что считалось нами как шероховатая поверхность.

  ... слайд 16 ...

  Прежде чем мы перейдем к рассмотрению результатов, необходимо указать некоторые шаги, которые были предприняты для ускорения симуляций. Во-первых, было решено воспользоваться новейшей CUDA-подобной технологией (C++ AMP), позволяющей с минимальными модификациями перенести вычисления алгоритма на GPU, поддерживающее DirectX-11. Не останавливаясь на технических деталях, скажу лишь что удалось достичь стократного прироста производительности над алгоритмом, выполняющися исключительно на CPU

  Однако, помимо чисто техниского вопроса ``на чем считать'', остается алгоритмический вопрос. Признаным стандартом является вычисление $10^{4 \div 5}$ шагов по времени, прежде чем снимать результаты симуляции. Не оспаривая потенциальной необходимости столь длительных вычислений для больших (порядка $10^6$ частиц) систем, для всех меньших масштабов это было сочтено преуеличением.

  ... слайд 17 ...

  Адресуя этот вопрос, нами был предложен алгоритм, позволяющий оценить, в зависимости от рассматриваемой задачи, или стабильность параметра порядка, или стабильность профиля скорости. Через каждые 100 итераций начинает снимаеться значение, соотвественно, средней скорости или дисперсии профиля скорости, усредняется по еще 20 шагам по времени, и сравнивается со значением с предыдущей итерации. Если расхождение не превышает $\frac{1}{\sqrt{N}}$, то для этого значения шума симуляция прекращается.

  ... слайд 18 ...

  Для того чтобы проверить правильность работы приложения вообще, и предложенного нами алгоритма в частности, была запущена симуляция с параметрами рассматриваемой области такими-же как в оригинальной работе Вичека и соавторов. Как видно из графика зависимости параметра порядка от безразмерного шума, наблюдается хорошее соответствие наших и ``классических'' результатов. Более острый резкий скачок фазового перехода связан с тем что в работе Вичека фигурирует усреднение по 5 конфигурациям, мы же, пользуясь случаем, провели усреднение по 20.

  ... слайд 19 ...

  Аналогочно предыдущему, была построена зависимость параметра порядка от безразмерного шума для задачи Куэтта, которая очевидно свидетельствует об увеличении шума, при котором происходит фазовый переход.

  ... слайд 20-22 ...

  Здесь представлены графики зависимости средней скорости от высоты над нижней (неподвижной) границей. Хотелось бы заметить, что не наблюдается никаких аналогий с профилем куэттовского потока ни для варианта с зеркальным отражением, ни для варианта с шероховатой нижней поверхностью

\end{document}