%!TEX root = ..\MainFile.tex
\section{Групповые явления в жизни животных} % (fold)
\label{sec:AnimalFlocking}
\epigraph{Если сто бегунов как один бегут,\\
Это можно назвать так и сяк.\\
У лошадей это будет табун,\\
У рыб это будет косяк.}{Машина Времени}
	\subsection{Методы наблюдения и сбора информации} % (fold)
	\label{sub:ExperimentalMethods}
	Главной сложностью в выполнении экспериментального наблюдения за группой животных или, тем более, насекомых, заключается в трудности отслеживания траекторий отдельных особей. Это связано с тем что
	\begin{itemize}
		\item рассматриваемые колонии или группы состоят из множества индивидов, которые
		\item выглядят очень похоже
		\item и в основном очень быстро перемещаются
	\end{itemize}
	Кроме того, необходимо учитывать крайнее разнообразие живых организмов, за перемещением которых необходимо наблюдать, поскольку, как уже говорилось во введении, изменение поведения при обьединении в группы наблюдается для широкого разнообразия типоразмеров животных - начиная от бактерий в пробирке \cite{csahok1997,keller1971} и заканчивая китами на океанских просторах \cite{makris2009}.
	Несмотря на совсем неочевидные методы решения этих проблем, изучению стадного поведения (flocking bechavior) на протяжении многих лет посвящалось значительное внимание. Можно заметить что особенно остро (хоть и не в такой формулировке) этот вопрос стоял в 20-е годы XIX столетия на территории Одеской области, которая подвергалась массовым нашествиям саранчи, однако без компьютерного оборудования или без достижения теории ренормализационных групп достичь каких-либо результатов было невозможно.

	Итак, возвращаясь к теме этого пункта, вкратце рассмотрим использующиеся сейчас методы регистрации траекторий индивидов в стаде (стае). Условно, их можно разделить на две неравные группы: в одну отнесем так или иначе методы с использованием визуальных средства (т.е. фото- и видео-камер, в комбинации с различного рода маячками или метками), а в другую сравнительно недавно начавшие применяться методы с использованием GPS-систем. И если применение методов из второй групы не вызывает никаких технических вопросов (кроме, очевидно, закрепления такого рода устройств на исследуемых индивидах), то на методах первой группы следует остановиться подробней.

	Классическим может считаться метод измерение скорости по изображениям частиц \cite{raffel2007}. Обычно он применяется для измерения скорости потока жидкости или газа: в исследуемую среду вводится краситель, состоящий из броуновских частиц, и наблюдая за перемещением этого вещества становится возможным определять линии тока в исследуемой среде. Применимо к самодвижущимся частицам метод успешно был применен Czir\'{o}k и соавторами\cite{csahok1997}, которые при помощи фазово-контрастной микроскопии исследовали групповое движение молекул.

	Когда речь заходит об исследовании стай птиц (стад животных), размер индивидов больше, однако в отличие от бактерий область перемещения слабо ограничена. Тогда движение записывается напрямую на камеру, с тем чтобы в дальнейшем восстановить отдельные траектории. К примеру, в 70-х годах аерофотосьемка использовалась для определения расположения отдельных индивидов во время сбивания африканских буйволов в стадо \cite{sinclair1977}. Однако использования одной камеры для фиксирования перемещений группы рыб или птиц явно недостаточно, поскольку как-то необходимо учитывать расстояние от камеры до особи. Чтобы обойти эту сложность, например, Бекко и соавторы \cite{becco2006} огриничили движение рыб по высоте используя плоский и широкий аквариум.

	Использование нескольких камер для сьемки стай с разных точек в дальнейшем требовало трудоемкой обработки результатов для получения трехмерной картины. Получение стереографических изображений с последующей обработкой позволило определить расстояния между соседними индивидуумами в стаях скворцов и чернозобиков \cite{major1978}. В менее давних исследованиях был обработан метод преобразования фотографий стаи в трехмерные коспьютерные модели, что позволило получить мгновенные снимки положения сотен скворцов в стае \cite{ballerini2008}, однако в задаче восстановление траекторий движения все еще остается поле для исследования.

	По мере развития технологий, стало возможным создать маячки GPS, достаточно маленькие, чтобы их можно было закрепить на животном или птице не мешая ему. К сожалению, с увеличением количества отслеживаемых индивидов возрастает и стоимость исследования. С другой стороны, такого рода устройства обладают ограниченной точностью в определении местоположения. Несмотря на приведенные выше ограничения, были проведены исследования пар \cite{biro2006,nagy2010}, или даже шестерок \cite{dellariccia2008} тренированных голубей.

	Помимо вышесказанного, в настоящее время увеличение точности сонаров и разработка определенной технологии применения, в которой океан выступает в роли акустического волновода позволило использовать их для единовременного наблюдения за огромными количествами рыб на больших территориях океанического шельфа. \cite{makris2006}

	Подводя итоги, можно отметить увеличение роли комьютера в накоплении и обработке экспериментальных данных. Текущая динамика позволяет надеяться что со временем будут получены данные не меньшей точности, чем получаемые во время численных экспериментов по различным моделям.
	% subsection subsection_name (end)

	\subsection{Основные результаты наблюдений} % (fold)
	\label{sub:ExperimentalResults}
	
	\ref{sub:ExperimentalResults}

	% subsection основные_результаты_наблюдений_за_системами_демонстрирующими_группувую_динамику (end)

% section групповые_явления_в_жизни_животных (end)